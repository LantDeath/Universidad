\documentclass[letterpaper]{article}

\title{Análisis de texto científico. }
\author{Liz Ángel Núñez Torres}

\usepackage{amsfonts}
\usepackage{amsmath}
\usepackage{xcolor}
\usepackage{float}
\usepackage{graphicx}
\usepackage{amssymb}
\usepackage{multicol}
\usepackage[spanish]{babel}
\usepackage{hyperref}
\usepackage{ragged2e}
\usepackage{ulem}
\usepackage{mathrsfs}
\usepackage{verbatim}


\begin{document}

\begin{titlepage}
    \maketitle
\end{titlepage}

\begin{center}
    
    \textbf{\huge Microscopic Origin of the Entropy of Astrophysical Black Holes}

\end{center}

\vspace{\baselineskip}

\vspace{\baselineskip}

\vspace{\baselineskip}

\begin{figure}[h]
    \centering
    \includegraphics[scale=1.0]{Ngero.PNG}
    \caption{  M87 - Abril de 2019 -  Event Horizon Telescope (EHT) }
    
\end{figure}

\textbf{Dirección o link del texto cientifíco:} \( \href{https://journals.aps.org/prl/abstract/10.1103/PhysRevLett.132.141501}{Aqui}\)

\newpage

\begin{justify}

Nuestro texto cientifíco tiene las siguientes caracteristicas:

\begin{itemize}    
    \item Encabezado con el nombre de la revista, número de texto publicado.
    \item Título.
    \item Autores.
    \item Instituciones enumeradas afiliadas a cada autor.
    \item Fecha de entrada, revisión, aceptado y publicación.
    \item Abstract o resumen.
    \item Introduccción.
    \item (\textit{Explicación del método}) Microestados de agujeros negros.
    \item Superposiciones gravitacionales cuánticas.
    \item Contando Microestados.
    \item Conclusión
    \item Bibliografía.
    \item Agradecimientos.
\end{itemize}
\end{justify}

\vspace{\baselineskip}

\begin{justify}

Los cinco primeros puntos podemos verlos en la siguiente imagén:

\end{justify}

\begin{figure}[h]
    \centering
    \includegraphics[scale=0.4]{Cuatro puntos.PNG}
    \caption{Physical Review Letters 132, 141501 (2024)}    
\end{figure}

\vspace{\baselineskip}

\begin{justify}

\textbf{1)} Vemos como en el inicio, exactamente en el encabezado, tenemos el nombre de la revista que ha publicado el texto junto con el número de texto cientifíco
y la fecha de publicación del mismo.
\end{justify}

\vspace{\baselineskip}

\begin{justify}
\textbf{2)} El título "\textbf{Microscopic Origin of the Entropy of Astrophysical Black Holes}" de su traducción "\textbf{Origen microscópico de la entropía de los \; \; agujeros negros astrofísicos}".
\end{justify}

\vspace{\baselineskip}
\begin{justify}
\textbf{3)} Autores del texto cientifíco:
\begin{itemize}
    \item \textbf{Vijay Balasubramanian.}
    \item \textbf{Albion Lawrence.}
    \item \textbf{Javier M. Magán.}
    \item \textbf{Martin Sasieta.}
\end{itemize}   
\end{justify}

\vspace{\baselineskip}

\begin{justify}

\textbf{4)} Podemos observar que cada autor está enumerado al igual que
las instituciones, esto es para facilitar el conocimiento sobre cada uno de los autores en
su participación en las diferentes instituciones.

\begin{itemize}
    \item David Rittenhouse Laboratory, University of Pennsylvania, 209 South 33rd Street, Philadelphia, Pennsylvania 19104, USA.
    \item Theoretische Natuurkunde, Vrije Universiteit Brussel, Pleinlaan 2, B-1050 Brussels, Belgium.
    \item Santa Fe Institute, 1399 Hyde Park Road, Santa Fe, New Mexico 87501, USA.
    \item Martin Fisher School of Physics, Brandeis University, Waltham, Massachusetts 02453, USA.
    \item Instituto Balseiro, Centro Atómico Bariloche, 8400-S.C. de Bariloche, Río Negro, Argentina.
\end{itemize}
\end{justify}

\vspace{\baselineskip}

\begin{justify}
\textbf{5)} Como se puede apreciar en la imágen, justo al inicio donde están las fechas
de entrada, revisión, aceptado y publicación hay un botón que nos lleva a CrossMark
donde podemos ver el estado del texto cientifíco.

\begin{multicols}{2}
\textbf{Fecha  entrada:} 9/01/2024.

\textbf{Fecha  revisión:} 31/01/2024. 
\end{multicols}

\begin{multicols}{2}
    \textbf{Fecha aceptado:} 9/02/2024.
    
    \textbf{Fecha publicación:} 1/04/2024. 
\end{multicols}
\end{justify}

\newpage
\begin{justify}

El abstract es el primer párrafo que encontramos.

\begin{figure}[h]
    \centering
    \includegraphics[scale=0.5]{Abstract.PNG}
    \caption{ Physical Review Letters 132, 141501 (2024)  }
    
\end{figure}
\end{justify}

\begin{justify}
En este párrafo, se encuentra resumido la investigación que se tratará a 
continuación en el texto. 
\end{justify}
\begin{itemize}
    \item " \(\)Construimos una familia infinita de microestados para agujeros negros en el espacio-tiempo de Minkowski que tienen
    descripciones semiclásicas efectivas en términos del colapso de capas de polvo en el interior del agujero negro. Cuántico
    Los agujeros de gusano mecánicos hacen que estos estados tengan superposiciones exponencialmente pequeñas, pero universales. Mostramos
    que estas superposiciones implican que los microestados abarcan un espacio de Hilbert de dimensión logarítmica igual al evento
    área del horizonte dividida por cuatro veces la constante de Newton, lo que explica el origen estadístico de la Bekenstein-
    Entropía de Hawking".
\end{itemize}
\vspace{\baselineskip}

\begin{justify}
\textbf{Introducción:}
\end{justify}

\begin{justify}
Nuestra Introduccción consta de cinco párrafos, en estos nos explican como trabajan, primero nos describen su fórmula,
\( S= \frac{A}{4G} \). Donde \(A\) es el horizonte de eventos, \(G\) es la constante
gravitacional de Newton. Trabajan con la constante de Planck y la velocidad de la luz es \textbf{1} dado que esta
fórmula funciona con todos los agujeros negros sin importar su momento ángular, masa o espacio de dimensión e intentan
dar una interpretación mécanica estadística de la entropía de los agujeros negros.
\end{justify}
\begin{justify}
¿Cuál es el origen de esta entropía? Esta pregunta fue resuelta por Strominger y Vafa, ellos explicaron la entropía de 
ciertos agujeros negros supersimétricos en términos del espacio de Hilbert subyacente a microestados de la teoría de cuerdas. 
Cálculos que fueron gracias a que los agujeros negros tienen múltiples tipos de cargas magnéticas y eléctricas
\end{justify}
\begin{justify}
¿Podemos tomar una explicación microscopica universal para la entropía de agujeros negros astrofísicos?
\end{justify}

\newpage

\begin{justify} 
La respuesta que nos dan es:
\end{justify}
\begin{justify}
<< Usamos el hecho de que  la mecánica estadística cuántica, cualquier superposición de microestados es también un microestado, donde un microestado
es un vector normalizable en el espacio de Hilbert con valores de expectativa fijos. Simplemente buscamos cualquier conjunto
de estados que sea lo suficientemente grande como para abarcar todo el espacio de Hilbert, este conjunto debe estar bajo control
para que podamos calcular una matriz de Gram de superposicionesde estados. El rango de la matriz de Gram determinan el número máximo
de microestados linealmente independientes.>>
\end{justify}

\vspace{\baselineskip}

\textit{\uline{A partir de aquí explicaremos los métodos planteados.}}

\begin{justify}
\textbf{Microestados de agujeros negros:}
\end{justify}
\begin{justify}
Para  este punto, han tenido que adaptar los microestados y los agujeros negros, generando una familia infinita de microestados y un agujero negro infinito.
Antes de generar los microestados, estos son definidos como un estado cuántico en el espacio de Hilbert de un agujero negro con valores fijos. Además, estos
microestados tendrán como característica soluciones semiclásicas de la relatividad general aclopados a la materia. Los agujeros tratados aquí no serán
formados desde el colapso, sino, siempre existentes siendo planos asintóticamente.
\end{justify}
\begin{justify}
El tiempo iniciaría desde el horizonte de eventos donde se efectua una singularidad de agujero blanco para posteriormente terminar en una singularidad de un agujero negro.
\end{justify}
\begin{justify}
Estos agujeros aunque no parten desde el colapso si comparten similitud geometría junto con el comportamiento tardio que se forman en los agujeros negros de colapso.
\end{justify}
\begin{justify}
Todos los estados contruidos comparten  entre el horizonte la misma geometría junto con el limite asintótico del espacio tiempo como son los microestados del agujero negro
vistos por un observador externo.
\end{justify}
\begin{justify}
Estos agujeros coinciden con los agujeros de Schwarzschild de radio: \[r_2=2GM.\] Donde \(M\) es \textbf{ADM}, una formulación Hamiltoniana de la relatividad general. Las geometrías
están imbricadas con materia diferente las cuales se van generando una tras otra distinta a la anterior.
\end{justify}
\begin{justify}
La materia por su parte está limitada a organizarce en capas esféricas de partículas de polvo con una resta de masa \(m\) de la capa anterior donde emerge de la singularidad pasada y se sumerge en la singularidad
futura sin dejar la región del agujero negro.
\end{justify}

\newpage

\begin{justify}
Para explicar esto, se han basado en el diagrama de Penrose, que podemos ver a continuación:
\vspace{\baselineskip}

\begin{figure}[h]
    \centering
    \includegraphics[scale=0.4]{Pnerose.PNG}
    \caption{Diagrama de Penrose de la evolución temporal de un microestado de un agujero negro eterno unilateral.}   
\end{figure}
\end{justify}
\begin{justify}
El estado semiclásico se define en el tiempo de reflexión simétrica en porción de Cauchy \(\Sigma\).
La geometría exterior se extiende entre los horizontes futuro y pasado \(\mathscr{H}^\pm \)
y los límites nulos conformes \(\mathscr{J}^\pm\) . El interior contiene una capa delgada \(W\), que divide la geometría entre una región de espacio plano menor dentro de la capa, y una región de geometría de agujero negro mayor fuera de la capa.
Las líneas en zigzag en la parte inferior y superior son el agujero blanco y las singularidades del agujero negro donde el tiempo comienza y termina. El estado semiclásico en \(\Sigma\) es no singular y perfectamente definido.
\end{justify}
\begin{justify}
La métrica exterior de los microestados es la habitual de Schwarschild.

\underline{(1)}
\[ds^2=-f\left(r\right)dt^2+\frac{dr^2}{f\left(r\right)}+r^2d\Omega^2.  \]
\end{justify}
\begin{justify}
Donde \(f\left(r\right)=1-r\left(r_s/r\right)\) y \(d\Omega^2=d\theta+sin^2\theta d\phi^2\) es la métrica redonda de la unidad esférica \textbf{S}\(^2\). 
\end{justify}
\begin{justify}
La métrica puede continuar hacia el interior del agujero negro siendo \(r\) menor que \(r_s\) y más de un segundo en la región asintota por la misma métrica.
Toda la geometría dentro y fuera de la capa están determinadas por la juntura de Israel que es usada para ajustar la métrica y espacio-tiempo a través de la capa.
Concretamente el volumen \(W\) de la capa que lleva localizada haciendo así un <<momentum>> y fluido perfecto sin presión.
\end{justify}
\begin{justify}
Obteniendo así: \[T_{\mu v}|_w =|\sigma u_\mu u_v \]
\end{justify}
\newpage
\begin{justify}
Donde \(\sigma\) es la densidad del fluido en la superficie y \(u^\mu\) es la cuarta velocidad del campo tangente a \(W\). El volumen de \(W\) es determinada por \(R\left(T\right)\)
el radio de la capa \(R\) como función adecuado a tiempo \(T\). La capa estará en \(r=R\left(T\right)\) con \(T\) determinada por el tiempo apropido a lo largo de toda la trayectoria de la capa.
\end{justify}
\begin{justify}
La junción de Israel utilizada es la de una partícula no relativista y de energia cero.
\end{justify}
\begin{justify}
Por lo tanto:\[R^2+V_{eff}\left(R\right)=0\].
\end{justify}
\begin{justify}
Donde se define el potencial efectivo:\[V_{eff}\left(R\right)=f\left(R\right)-\left(\frac{M}{m}-\frac{Gm}{2R}\right)^2.\]
\end{justify}
\begin{justify}
Siempre que \(M\) sea menor o igual a  \(m\), la capa se expadirá desde la singularidad del pasado, localizada en el tiempo finito adecuado en el pasado e iniciando la segunda región donde alcanza el máximo
radio siendo \(R_*\) mayor o igual a \(r_s\) la capa re colapsa en un radio \(R_* \approx Gm/2\) debido a su  energía gravitacional.
\end{justify}
\begin{justify}
La geometría dentro de la capa consiste de una porción del espacio plano \; \; (Minkoswki):\[ds^2_< =  -d\tilde{t}^2 + d\tilde{r}^2 + \tilde{r}^2 d\Omega^2\]
\end{justify}
\begin{justify}
Esta cubre suavemente en \(\tilde{r}=0\). Afuera la geometría de dos caras está cortada en \(r=R_* \leq r_s\) en la parte izquierda.
\end{justify}
\begin{justify}
La geometría inducida rebanada se asemeja a un <<Wheeler's bag of god>> cuando la masa en la capa es más grande \(m>>M\) en un volumen de \(\Sigma _{in} \subset \Sigma\)
de escalas como el volumen \(\left(\Sigma\right)\approx \left(\pi/3\right)\left(Gm\right)^2 \).
\end{justify}
\vspace{\baselineskip}
\begin{justify}
\textbf{Superposiciones gravitacionales cuánticas:}
\end{justify}
\begin{justify}
En este punto abarcarán  hipersuperficies, geometrías Euclideanas en espacios infinitos junto con estados semiclásicos asociados a microestados en el agujero negro. La descripción de estos últimos son construidos sobre el tiempo de reflexión simétrica
donde \(t=o\) cortada por \(\Sigma\) a través del camino integral Euclideo.
\end{justify}
\begin{justify}
El estado en el que trabajan, homólogo al \textbf{Estado Hartle-Hawking} está definido en datos iniciales regulares que concomitantemente van determinando la evolución del \textbf{tiempo de Lorentzian} hacia el futuro
y el pasado. 
\end{justify}
\begin{justify}
Recordemos que se trabaja con estados y microestados infinitos del espacio de Hilbert donde \(m\) determina la masa adecuada a la insercción de materia correspondiente en el interior del agujero negro. No existe limite superior a \(m\).
\end{justify}
\begin{justify}
Esta familia infinita esta inocentemente sobre la \textbf{Entropía Bekenstein-Hawking} pero esto es únicamente si los estados son ortogonales entre si. 
\end{justify}
\begin{justify}
Usando así una simple descripción efectiva de los microestados en terminos de acción gravitacional Euclideana aclopada a una capa delgada llegando a:
\end{justify}
\begin{justify}
\[I[X]=-\frac{1}{16\pi G} \int_{x}^{} R+\frac{1}{8\pi G}\int_{\partial X}^{}K+\int_{W}^{}\sigma+I_{ct}. \]
\end{justify}
\begin{justify}
Donde \(R\) es el escalar de Ricci de la variedad Euclideana \(X\), \(K\) es la curvatura extrínseca de su limite \(X\), \(\partial\) es la densidad de la capa, \(W\) es el volumen  mundial de la capa y por último \(I_{ct}\) corresponde
al contratermino de substracción de fondo que elimina divergencias y normaliza el valor de la enacción fantasma.
\end{justify}
\begin{justify}
El solapamiento que viene de la variedad Euclideana que posee una sola asintota donde la trayectoria empieza y termina en \(m=m'\) posteriormente creadas se pueden ver en la siguiente gráfica:
\end{justify}
\begin{justify}
\begin{figure}[h]
    \centering
    \includegraphics[scale=0.4]{Euclideana.PNG}
    \caption{Continuación euclidiana de la geometría espacio-temporal de los microestados a lo largo de \(\Sigma\) }   
\end{figure}
\end{justify}
\begin{justify}
La sección euclidiana consiste en un agujero negro euclidiano (derecha), y una región de un espacio plano euclidiano (izquierda), pegados juntos a lo largo de la trayectoria de una capa delgada. La cáscara comienza en el infinito espacial asintótico,
se recupera en \(R_*\) y vuelve a \(R= \infty\). Los tiempos euclidianos \(\tilde{\beta}_m , \tilde{\beta '}_{m'} \leq \beta\) dependen de la masa de la capa.
\end{justify}
\begin{justify}
El solapamiento entonces va encausado de forma:

\[\left< \Psi_{m}|\Psi_{m'} \right>= \delta_{mm^'} \]
\end{justify}
\begin{justify}
Donde la sobre barra la cual he omitido por errores de programación, denotaría  que el cálculo se hace con las reglas de integral de la trayectoria gravitacional y donde han normalizado
los estados usando la acción de la capa en la variedad Euclideana correspondientes \(Z^{(m)}_1 = e^{-I[X_m]}\).
\end{justify}
\begin{justify}
A simple vista parecen ser ortogonales  pareciendo así que estos están en un espacio infinito de Hilbert, sin embargo esto es cambiado por las contribuciones
del agujero de gusano semiclásico de ruta integral a la trayectoria Euclideana. Estos agujero de gusano corresponden a efectos no perturbativos en gravedad cuántica
\end{justify}
\begin{justify}
Es entonces que al empezar con un agujero de gusano con dos limites explicitos a la esquina del solapamiento se obtiene:
\vspace{\baselineskip}
\vspace{\baselineskip}

\; \; \; \; \; \; \; \; \; \; \; \; \; \(\left< \Psi_{m}|\Psi_{m'} \right>= \delta_{mm^'}\) \( + \frac{Z_2}{Z_1^{(m)}Z_1^{(m')}} \).
\end{justify}
\vspace{\baselineskip}
\begin{justify}
La nueva contribución viene dada por \(Z_2=e^{-[X_2]}\) donde \(X_2\) es el multiple agujero de gusano Euclideano que se extiende en dos limites asintóticos preparando así 
cada superposición de un orden \(\textit{O}\left(e^{-s}\right)\) contribuciones, de la cuales cualquier termino dependiente de la masa en la contribución
sin agujero de gusano. Esta solución está construida cortanbdo y pegando dos soluciones de agujeros negros Euclideanos a lo largo de la trayectoria de las dos capas delgadas.
\end{justify}
\begin{justify}
Para terminar este punto, vamos a ver una gráfica donde podemos ubicar con facilidad el solapamiento.

\begin{figure}[h]
    \centering
    \includegraphics[scale=0.4]{Solapamiento.PNG}
    \caption{Contribución del agujero de gusano euclidiano al segundo movimiento de la superposición. El agujero de gusano tiene los dos productos internos como sus límites. Consta de dos agujeros negros euclidianos en el espacio plano, pegados a lo largo de las dos conchas. }   
\end{figure}
\end{justify}

\newpage
\begin{justify}
\textbf{Contando Microestados.}
\end{justify}
\begin{justify}
En este punto se genera otro supuesto, una subfamilia de microestados de agujeros negros \(\left\{ |\Psi_{mj}>  \right\}\) para capas con masas
\(m_j = jm\) para \(j=1,2,,,\) donde \(m\) es un valor suficientemente grande de la masa. Esto se genera para determinar el espacio de Hilbert atravesado por los microestados considerando
la matriz de superposiciones de Gram para un subconjunto de esos estados. \[G_{ij}=\left< \Psi_m || \Psi_{m'} \right>.\] 
Donde \(i,j=1\). 
\end{justify}
\begin{justify}
La matriz presentada es Hermitica de Gram semidefinida positivamente, construyendo así microestados independientes en un subconjunto equivalente al rango de la matriz en la dimensión espacial
de Hilbert abarcada por el conjunto donde se albergan estos subconjuntos, conjuntos y microestados.
\end{justify}
\begin{justify}
En esta matriz de Gram, los cálculos sugieren una interpretación como una matriz aleatoria con momentos dados por las superposiciones universales.
\end{justify}
\begin{justify}
Estas expresiones universales se degeneran microcanonicamente en expresiones anteriores en una energia dada a traves de una tranformada de Laplace inversa de las constribuciones de agujeros negros.
\end{justify}
\begin{justify}
Donde obtienen: \[Z_{bh}(n\beta)=\int_{}^{} dE\rho(E)e^{-n\beta E}.\]
Y para una ventana microcanonica dasa las energias  \([E,E+\Delta E]\) de donde se definen las
funciones: \[e^{\textbf{S}}\equiv \rho (E)\Delta E, \; \; \; \textbf{Z}_n \equiv \rho (E)e^{-n\beta E}\Delta E.\]
Con todo esto definido, logran llegar a tener una coincidencia en \(\textbf{S}\) con la entropía de Bekenstein-Hawking.
\[\textbf{S}=\frac{A}{4G}.\] Es en este punto donde se recontextualiza  como una función que controla las contribuciones del agujero de gusano, donde pasan a describir el comportamiento del mismo
concluyendo que el rango de la matriz de Gram es el número de valores propios contenidos en la parte continua de la distribución, obteniendo así, que el rango de \(G\) es dado por \(e^{\textbf{S}}.\)
\end{justify}
\newpage
\begin{justify}
\textbf{Conclusión:}
Para concluir, los autores se encuentran conque al añadir microestados, hay un punto en el que los estados no pueden ser ortogonales, esto está controlado por las estadisticas universales del producto interno
por la entropía de Bekensteins-Hawking.
\end{justify}
\begin{justify}
De manera que el problema del origen microscópico de la entropía es contar cuantos estados ortogonales se pueden contruir a partir de los conjuntos de microestados. Este conteo da lugar a la dimensión Bekenstein-Hawking.
\end{justify}
\begin{justify}
Los resultados entonces llegan a explicar la entropía del agujero negro en una teoría que tiene aclopada la relatividad general a la materia masiva como un limite de baja energia.
\end{justify}
\begin{justify}
Cabe resaltar que no se utilizaron detalles sobre Ads/CFT o teoría de cuerdas.
\end{justify}

\vspace{\baselineskip}

\begin{justify}
\textbf{Bibliografía:}
\end{justify}
\begin{justify}
Para este texto cientifíco se utilizaron dicisiete textos más, de los cuales enumeraremos solo cinco de estos.
\end{justify}
\begin{justify}
\begin{itemize}
    \item J. Maldacena, S. H. Shenker, and D. Stanford, A bound on
    chaos, J. High Energy Phys. 08 (2016) 106.
    \item  J. D. Bekenstein, Black holes and entropy, Phys. Rev. D 7,
    2333 (1973).
    \item S. W. Hawking, Particle creation by black holes, Commun.
    Math. Phys. 43, 199 (1975); 46, 206(E) (1976).
    \item W. Israel, Singular hypersurfaces and thin shells in general
    relativity, Nuovo Cimento B 44, 1 (1966).
    \item Ioanna Kourkoulou and Juan Maldacena, Pure states in
    the SYK model and nearly-AdS2 gravity, arXiv:1707
    .02325.
\end{itemize}
\vspace{\baselineskip}

\end{justify}
\begin{justify}
\textbf{Agradecimientos:}
\end{justify}
\begin{justify}
Esta es la única parte subjetiva del texto en la cual podemos presenciar cada persona que indirectamente contribuyo a elucubrar y desarrollar este texto cientifíco,
personas como José Barbón, conocido físico en España.
\end{justify}
\begin{justify}
Dejo aquí el mensaje de los autores:
\end{justify}
\begin{justify}

<<Nos gustaría agradecer a José Barbón, Alex Belin, Jan de Boer, Horacio Casini, Kristan Jensen, Henry Maxfield y Roberto Emparan por sus útiles discusiones. V. B. y J. M. son apoyados en parte por el Departamento de Energía a través de DE-SC0013528 y QuantISED DE-SC0020360, así como por la Fundación Simons
Foundation a través de It From Qubit Collaboration (Subvención núm. 38559). A. L. y M. S. son apoyados en parte por el Departamento de Energía a través de DE-SC0009986 y QuantISED DE-SC0020360. A este preprint se le asigna el código BRX-TH-6713>>.





\end{justify}

\end{document}