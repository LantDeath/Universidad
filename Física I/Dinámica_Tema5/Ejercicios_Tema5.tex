\documentclass[a4paper,12pt]{article}

% Paquetes esenciales
\usepackage[utf8]{inputenc}      
\usepackage[T1]{fontenc}        
\usepackage[spanish]{babel}      
\usepackage{ragged2e}
\usepackage{multicol}
\usepackage{wasysym}
\usepackage{enumitem}
\usepackage{amsmath, amssymb}    
\usepackage{amsfonts}            
\usepackage{geometry}            
\geometry{left=2.5cm, right=2.5cm, top=2.5cm, bottom=2.5cm}
\usepackage{graphicx}            
\usepackage{hyperref}            
\hypersetup{
    colorlinks=true,       % Activa los enlaces de colores
    linkcolor=blue,        % Color para enlaces internos
    urlcolor=cyan,         % Color para URLs
    citecolor=red          % Color para citas
}


\begin{document}

\title{Ejercicios Tema 5}
\author{Liz' Ángel Núñez }
\date{\today} 

\maketitle

\section{Calcular aceleración de la gravedad en la tierra.}
\vspace{\baselineskip}
\begin{justify}
Tenemos los datos:
\end{justify}

% Usamos los "\;" porque el comando \begin{colums}{n} se va a la puta :p.
    \(r = 6378.137 km.\)\;\;\;\;\;\;\;\;\;\;\;\;\;\; \(\mu   = G M  = 3.986004418 \times 10^{14} m^3 s^{-2}.\)

\vspace{\baselineskip}

\begin{justify}
    Usaremos las formulas de fuerza y gravitación:
\end{justify}
\begin{multicols}{2}
    \(F=G\;\frac{m_1 m_2}{r^2}.\)

    \(F = mg. \)
\end{multicols}
\vspace{\baselineskip}
\begin{justify}
    Reemplazamos la fuerza por su equivalencia en la fórmula de la gravitación:
    \vspace{\baselineskip}

\(mg = G \; \frac{M  m_2}{r^2}\) \(\rightarrow\) \(g =\frac{G M}{r^2} \) \(\rightarrow\) \(g =\frac{\mu }{r^2}.\)
\end{justify}
\vspace{\baselineskip}

\begin{justify}
    Usamos los datos suministrados al inicio y obtenemos el resultado:
\end{justify}
\vspace{\baselineskip}

\begin{center}
    \(g =\frac{3.986004418 \times 10^{14} \frac{m^3}{s^2}}{6378137^2 m^2} = 9,8\; m\;  s^{-2}.\)    
\end{center}
\vspace{\baselineskip}

\section{Repetir los cálculos del problema 1 para el sol en 1UA.}
\vspace{\baselineskip}

\begin{justify}
    Contamos con los datos:
\end{justify}

% Usamos los "\;" porque el comando \begin{colums}{n} se va a la puta :p.
    \(1AU = 1.495978707 \times 10^{11} m. \) \;\;\;\;\;\;\;\;  \(\mu  = GM  =  1.32712440018 \times 10^{20} m^3 s^{2} \)

\vspace{\baselineskip}

\begin{justify}
    Hacemos el mismo razonamiento que en el punto anterior despejando la aceleración "g" de la fuerza en la fórmula de la gravitación y obtenemos la expresión:
\end{justify}


\begin{center}
    \(g =\frac{\mu }{r^2}.\)
\end{center}

\vspace{\baselineskip}

\begin{justify}
    Por lo tanto obtenemos:
\end{justify}

\begin{center}
    \(g = \frac{1.32712440018 \times 10^{20} m^3 s^{-2}}{(1.495978707 \times 10^{11})^2 m^2} = 5.93 \cdot 10^{-10} m\;s^{-2}\)
\end{center}

\begin{justify}
    Comparando el resultado de la aceleración de la gravedad en la tierra y el sol a una distancia 1UA podemos notar que la
    aceleración del sol es menor que la de la tierra. De este resultado se puede argüir que la diferencia radica en la dependencia inversa
    del cuadrado de la distancia. También podemos decir que es la gran masa del sol lo que logra mantener la localidad de los otros astros.
\end{justify}

\vspace{\baselineskip}

\section{Calcular la aceleración de una nave desde el ecuador y la distancia  en la cual podemos dejar en órbita el satélite.}
\begin{justify}
    Tenemos los datos:
\end{justify}
\begin{itemize}
    \item Día sidéreo  \(\rightarrow \;23h \; 56m \; 4091s\).
    \item Radio terrestre \(\rightarrow \; r  = 6378.137 km.\)
    \item Parámetro gravitacional estándar \(\rightarrow \;\mu   = G\cdot M  = 3.986004418 \times 10^{14} m^3\;s^{-2}.\)
\end{itemize}
\vspace{\baselineskip}

\begin{justify}
    Usaremos las fórmulas:
\end{justify}

% Usamos los "\;" porque el comando \begin{colums}{n} se va a la puta :p.
\(a_c = \frac{v^2}{r}.\) \;\;\;\;\;\;\;\;\; \(F = G \cdot \frac{m_1\cdot m_2}{r^2}.\) \;\;\,\;\;\;\;\;\; \(F_c = m \cdot a .\)

\vspace{\baselineskip}
\begin{justify}
    Obtenemos el periodo órbital:
\end{justify}
\(T = 82800 + 3360 + 4,091 \; \rightarrow T = 86164.091\).

\begin{justify}
    Tenemos que la velocidad tangencial es \(v_t = \frac{2\pi r}{T}\) que usaremos en nuestra fórmula \(a_c = \frac{v_t^2}{r} \rightarrow a_c =  \frac{(\frac{2\pi r}{T})^2}{r} \rightarrow \frac{4\pi^2 r^2}{T^2 r} \rightarrow \frac{4\pi^2 r}{T^2}.\)
\end{justify}
\begin{justify}
Y \(F = G \cdot \frac{m_1\cdot m_2}{r^2} \rightarrow F_c = m \cdot a . \rightarrow m\cdot a = G \cdot \frac{m_1\cdot m_2}{r^2} \rightarrow  a_c =  \frac{G\cdot M  }{r^2} \rightarrow a_c = \frac{\mu }{r^2}.\)
\end{justify}
\vspace{\baselineskip}
\begin{justify}
    Igualaremos la fuerza gravitacional \(\left(\frac{\mu }{r^2}\right)\) con la aceleración centrípeta \(\left(a_c = \frac{4\pi^2 r}{T^2}\right)\) porque la fuerza gravitatoria es la causa de la acelearación centrípeta.
\end{justify}
\vspace{\baselineskip}

\(a_c = \frac{\mu }{r^2} = \frac{4\pi^2 r}{T^2}.\) 
\vspace{\baselineskip}


\(\mu T^2 = 4\pi^2 r^3 \rightarrow  \frac{\mu T^2}{4\pi^2}=r^3 \rightarrow r = \sqrt[3]{\frac{\mu T^2}{4\pi^2}}. \)
\vspace{\baselineskip}

\begin{justify}
    Operamos: 
\end{justify}

\(r^3 = \frac{3,986004418 \times 10^{14}\left(86164091\right)^2}{4\pi^2} \Rightarrow r = \sqrt[3]{7.496018685512828\times 10^{22}} = 4.2164\times 10^{6}m.  \)
\vspace{\baselineskip}

\begin{justify}
    Como lanzamos nuestro satélite desde la superficie de la tierra sobre el ecuador, debemos restarle a nuestro resultado el radio de la tierra.
\end{justify}
\begin{center}
    
\(4.2164\times 10^{6}m - 6378137m = 35781.863 km\).
\end{center}

\begin{justify}
    Como hemos visto en nuestro procedimiento, masa del satélite no es relevante en los cálculos, por ende esta no afectará la orbita del mismo si esta se triplicase.
\end{justify}


\section{Calcula a qué distancia de la superficie de Marte, desde el ecuador, ¿Podemos orbitar manteniendo la misma ubicación para un espectador desde su superficie?.}

\vspace{\baselineskip}

\begin{justify}
    Datos:
\end{justify}
\begin{itemize}
    \item Día sidéreo  \(\rightarrow \;24h \; 37m \; 22.663s\).
    \item Radio Marte \(\rightarrow r \male = 3389.5 km\).
    \item Parámetro gravitacional estándar \(\Rightarrow \mu\male = GM\male = 4.282837 \times 10^{13} m^3\cdot s^{-2}\).
\end{itemize}

\vspace{\baselineskip}
\begin{justify}
    Obtenemos el periodo órbital
\end{justify}

\(T = 86400 + 2220 +22.663 = 88,642.663 s\).
\vspace{\baselineskip}

\begin{justify}
    Al igual que en punto pasado, aplicaremos el mismo razonamiento para llegar a un cálculo igual con diferentes variables.
\end{justify}
\vspace{\baselineskip}

\(r^3 = \frac{\mu T^2}{4\pi^2} \rightarrow r \sqrt[3]{\frac{\mu T^2}{4\pi^2}}\).

\begin{justify}
    Usamos los datos:
\end{justify}
\(r^3 = \frac{4.282837 \times 10^{13}\left(88,642.663\right)^2}{4\pi^2} = \sqrt[3]{8.723307300852058 \times 10^{27}} = 20585451m\)

\begin{justify}
    Restaremos el radio de marte a nuestro resultado.
\end{justify}
\vspace{\baselineskip}

\(20585451 - 3389500 = 17195951 \Rightarrow 17195.951 km.\)

\section{¿Qué aceleración sentirá un astronauta en el despegue debido a las fuerzas de ligadura producidas por el asiento?}
\vspace{\baselineskip}

\begin{justify}
    Datos:
\end{justify}

\begin{itemize}
    \item \(5000 t \Rightarrow  5.000.000 kg\) .
    \item \(74 MN \Rightarrow 74.000.000 n\).
\end{itemize}
\vspace{\baselineskip}

\begin{justify}
    Para saber la aceleración del cohete usaremos la segunda ley de Newton \(F = m \cdot a \)
    donde al despejar la aceleración \(a = \frac{F}{m}\).
\end{justify}

\begin{justify}
    Tendriamos:
\end{justify}

\(a = \frac{74 \times 10^6}{5 \times 10^6} = 14.8 m \cdot s^{-2}\).

\begin{justify}
    Sin embargo, para saber la aceleración neta que experimenta el astronauta debemos añadir tambien la gravedad, es entonces que decimos que \(a_t = a + g.\)
    Donde \(a_t\) es la aceleración total.
\end{justify}

\vspace{\baselineskip}
\(a_t = 14.8 \; m \cdot s^{-2} + 9.81 \; m \cdot s^{-2} = 24.61 \; m \cdot s^{-2} .\)

\begin{justify}
    Donde el peso relativo que experimenta el astronauta es \(P_r = \frac{a_t}{g} \rightarrow \frac{24.61}{9.81}=2.50\).
\end{justify}

\begin{justify}
Podemos afirmar que el astronauta siente un aceleración \(2.51\) veces la gravedad de la tierra.
\end{justify}

\vspace{\baselineskip}
\section{Calcular el coeficiente mínimo de fricción estática \(\mu_s\). }
\vspace{\baselineskip}

\begin{justify}
    Datos:
\end{justify}

\begin{itemize}
    \item \(v = 36 \; km \cdot h^{-1} \Rightarrow \frac{36 \cdot 1000}{3600} = 10 \; m \cdot s^{-1}.\)
    \item \(r = 15 \; m\).
\end{itemize}

\vspace{\baselineskip}

\begin{justify}
    Para determinar el coeficiente de fricción estática usaremos la primera y segunda ley de Newton.
\end{justify}

\begin{enumerate}
    \item \(\Sigma F_y = 0 \Rightarrow\) \;\;\;\;\;\;\;\;\;\;\;\;\;\;\;\;\;\;\;\; Primera ley para el eje vertical \(\left(y\right)\).
    \item \(\Sigma F_x = m \cdot a \Rightarrow\) \;\;\;\;\;\;\;\;\;\;\;\;\;\; Segunda ley para el eje horizontal \(\left(x\right)\).
\end{enumerate}

\vspace{\baselineskip}
\begin{justify}
    La fuerza que interviene en el eje horizontal es la fuerza de fricción \(\left(F_f\right)\) que es equivalente a masa por aceleración centrípeta \(\left(a_c\right)\).
\end{justify}

\(F_f= m\cdot a_c\).

\begin{justify}
    Por definición la fuerza de fricción es igual al producto de \(\left(\mu\right)\) por la normal \(\left(N\right)\), y la aceleración centrípeta al cociente del cuadrado de la velocidad entre el radio \(\left(\frac{v^2}{r}\right)\).
\end{justify}

\(\mu \cdot N = m \cdot \frac{v^2}{r}\).

\begin{justify}
    Como no tenemos el valor de la normal \(\left(N\right)\), usaremos la primera ley de Newton, las fuerzas que intervienen en el coche verticalmente son la normal y el peso \(\left(m \cdot g\right)\).
\end{justify}

\(N - m \cdot g = 0 \Rightarrow N = m \cdot g \).

\begin{justify}
Llegamos a la siguiente conclusión:
\end{justify}

\(\mu \cdot m \cdot g = m \frac{v^2}{r} \;\;\; \rightarrow \;\;\; \mu \cdot g = \frac{v^2}{r} \;\;\; \Rightarrow \;\;\; \mu = \frac{v^2}{g \cdot r} .\)

\begin{justify}
    Sustituyendo por nuestros datos obtendremos el coeficiente mínimo de fricción estática.
\end{justify}

\(\mu_s = \frac{10^2}{9.81 \cdot 15} = 0.6795.\)

\vspace{\baselineskip}















































\end{document}