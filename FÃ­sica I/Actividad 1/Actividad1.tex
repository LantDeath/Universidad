\documentclass[a4paper,12pt]{article} % Tipo de documento y tamaño de letra
\usepackage[utf8]{inputenc} % Para que reconozca caracteres especiales como tildes
\usepackage[spanish]{babel} % Para definir el idioma y el formato de fechas en español
\usepackage{amsmath, amssymb} % Paquetes para símbolos y ecuaciones avanzadas
\usepackage{ragged2e}



\title{Actividad 1} 
\author{Liz Ángel Núñez Torres} 
\date{\today} 

\begin{document}

\maketitle 


\section{Ejercicio.} % Comienza una nueva sección

\begin{justify}
Realiza la sustitución de la ecuación 2 en la ecuación 3 para llegar a obtener una expresión de \(\theta(r)\).
Una vez conseguida la ecuación de \(\theta(r)\), despeja r en función de \(\theta\), para obtener así r.
\end{justify}
\begin{justify}
Continua ahora empleando la ecuación 1, para obtener una expresión de \(r(\theta)\)
dependiente de las variables iniciales del problema \(v_1,v_2,t_c\).
\end{justify}
\begin{justify}
Esta última ecuación que has obtenido (ecuación 4) será la que empleemos más adelante para dibujar gráficamente la parte no lineal del recorrido
\end{justify}

\vspace{\baselineskip}

\begin{enumerate}
    \item \( D = \frac{v_1v_2t_c}{v_2-v_1} \)
    \item \(r\left(t\right)=v_1\left(t-t_i\right) + D \)
    \item \(\theta\left(t\right)=\sqrt{\left(\frac{v_2}{v_1}\right)^2 -1} \; ln\left(1+\frac{v_1\left(t-t_i\right)}{D}\right)\)
\end{enumerate}

\vspace{\baselineskip}

\section{ Expresión \(\theta\left(r\right)\)}

\vspace{\baselineskip}

\begin{justify}
    Tomaremos la \textit{ecuación 2} para obtener la expresiónque se adecua en nuestra \textit{ecuación 3}.
\end{justify}


\vspace{\baselineskip}

\(r\left(t\right)=v_1\left(t-t_i\right)+D\)  \( \Rightarrow \) \( \frac{r(t)}{D} = \frac{v_1\left(t-t_i\right)}{D} + \frac{D}{D} \) \( \Rightarrow \) \(\frac{r(t)}{D} - 1 = \frac{v_1\left(t-t_i\right)}{D} \)

\vspace{\baselineskip}

\begin{justify}
Obteniendo la siguiente expresión:

\[
\boxed{\frac{r(t)}{D} - 1 = \frac{v_1\left(t-t_i\right)}{D}}
\]
\end{justify}

\vspace{\baselineskip}
\begin{justify}
Que sustituiremos en la \textit{ecuación 3} para obtener \(\theta(r)\).
\end{justify}

\vspace{\baselineskip}

\(\theta\left(t\right)=\sqrt{\left(\frac{v_2}{v_1}\right)^2 -1} \; ln\left(1+\frac{r(t)}{D} - 1\right)\) \(\Rightarrow\) \(\theta\left(t\right)=\sqrt{\left(\frac{v_2}{v_1}\right)^2 -1} \; ln\left(\frac{r(t)}{D}\right)\)

\vspace{\baselineskip}

\section{Despejar \( r \) en función de \(\theta\).}

\vspace{\baselineskip}
\(\theta=\sqrt{\left(\frac{v_2}{v_1}\right)^2 -1} \; ln\left(\frac{r}{D}\right)\) \(\Rightarrow\) \(\frac{\theta}{\sqrt{\left(\frac{v_2}{v_1}\right)^2 -1}} = ln\left(\frac{r}{D}\right) \) \( \Rightarrow \) \(e^{\frac{\theta}{\sqrt{\left(\frac{v_2}{v_1}\right)^2 -1}}} = \frac{r}{D} \)

\vspace{\baselineskip}

\begin{justify}
    Llegando a la siguiente ecuación:
\end{justify}

\[
\boxed{r = D \cdot e^{\frac{\theta}{\sqrt{\left(\frac{v_2}{v_1}\right)^2 -1}}} }
\]


\section{Expresión de r(\(\theta\)). }

\vspace{\baselineskip}

\begin{justify}
    Para este punto, recordamos la \textit{ecuación 1} y la \textit{ecuación r}.
\end{justify}
\begin{itemize}
    \item \( D = \frac{v_1v_2t_c}{v_2-v_1} \)
    \item \(r = D \cdot e^{\frac{\theta}{\sqrt{\left(\frac{v_2}{v_1}\right)^2 -1}}} \)
\end{itemize}

\begin{justify}
    Como podemos observar, podemos sustituir y obtener \( r(\theta) \) en términos de las variables iniciales de la \textit{ecuación 1}.
\end{justify}
\vspace{\baselineskip}

\[
\boxed{r(\theta) = \frac{v_1v_2t_c}{v_2-v_1} \cdot e^{\frac{\theta}{\sqrt{\left(\frac{v_2}{v_1}\right)^2 -1}}} }
\]


\end{document}

