\documentclass[letterpaper]{article}

\title{LABORATORIO MATLAB }
\author{Liz Ángel Núñez Torres}

\usepackage{amsfonts}
\usepackage{amsmath}
\usepackage{xcolor}
\usepackage{float}
\usepackage{graphicx}
\usepackage{amssymb}
\usepackage{multicol}
\usepackage[spanish]{babel}
\usepackage{hyperref}
\usepackage{ragged2e}
\usepackage{ulem}
\usepackage{mathrsfs}
\usepackage{verbatim}
\usepackage{listings}

\begin{document}

\begin{titlepage}
    \maketitle
\end{titlepage}

\begin{justify}
    \textbf{Preparación de datos y funciones.}
\end{justify}
\begin{enumerate}
    \item Definir la función a explorar.
    \item Definir el conjunto de \(x\) a definir.
    \item Utiliza MATLAB para definir funciones.
\end{enumerate}

\vspace{\baselineskip}

\begin{justify}
\textbf{Definir funciones:}    
\end{justify}
\begin{justify}
    Hemos definido las siguientes funciones en nuestro editor:
\end{justify}
\begin{itemize}
    \item " trigo "  donde calcularemos el seno y coseno de los valores que queramos analizar. 
\end{itemize}
\begin{figure}[H]
    \centering
    \includegraphics[scale=0.6]{trigo.PNG}
    \caption{Función \textbf{trigo} Tomada de MATLAB }    
\end{figure}
\begin{itemize}
    \item " miFuncion " función donde tenemos un polinomio. 
\end{itemize}
\begin{figure}[H]
    \centering
    \includegraphics[scale=0.6]{miFuncion.PNG}
    \caption{Función \textbf{miFuncion} Tomada de MATLAB }    
\end{figure}
\newpage
\begin{itemize}
    \item " Lfuncion " función donde tenemos un nuevo polinomio. 
\end{itemize}
\begin{figure}[H]
    \centering
    \includegraphics[scale=0.6]{Lfuncion.PNG}
    \caption{Función \textbf{Lfuncion} Tomada de MATLAB }    
\end{figure}

\vspace{\baselineskip}

\begin{justify}
    Ahora que hemos definido nuestras funciones, vamos a evaluarlas con diferentes valores.
\end{justify}
\begin{figure}[H]
    \centering
    \includegraphics[scale=0.6]{Valores.PNG}        
\end{figure}
\begin{justify}
    Como podemos apreciar, en la función \textit{" miFuncion "} hemos evaluado el seno y coseno con la función \textit{ " trigo "} para valores de \(x=5\).
\end{justify}
\begin{justify}
   Después hemos evaluado en nuestra función \textit{" Lfuncion "} con valores de \(x = 3\).
\end{justify}

\vspace{\baselineskip}

\textbf{Cálculos Básicos.}
\begin{enumerate}
    \item Utiliza las funciones definidas y las funciones apropiadas de Matlab para
    calcular derivadas y sus gráficas, la integral en un rango del dominio de
    terminado, el área de la región acotada por la gráfica seleccionada, el eje \(y=0\)
    y dos valores apropiados \(x_0\),\(x_1\) que acoten la región.

    \item Utiliza las funciones de MATLAB, derivadas (diff) de diversos órdenes,
    para encontrar la segunda y tercera derivada de la función seleccionada.

    \item Verifica los resultados con cálculos a mano o utilizando herramientas de
    cálculo simbólico si es posible.
\end{enumerate}

\vspace{\baselineskip}

\begin{itemize}
    \item Vamos a desarrollar nuestro primer punto.
\end{itemize}
\begin{figure}[H]
    \centering
    \includegraphics[scale=0.6]{deriv.PNG}        
\end{figure}
\begin{justify}
    Hemos calculado las derivadas de nuestras dos funciones \textit{" miFuncion "} y \textit{" Lfuncion "} para a continuación
    definir estas derivadas.
\end{justify}
\begin{multicols}{2}
    \textbf{Función:}

    \textbf{Derivada:}
\end{multicols}
\begin{multicols}{2}
    \textit{miFuncion}

    \textit{miDerivada}   
\end{multicols}
\begin{multicols}{2}
    \textit{Lfuncion}

    \textit{Lderivada}   
\end{multicols}

\vspace{\baselineskip}
\newpage

\begin{justify}
    Después de identificar nuestras derivadas como \textit{" miDerivada "} y \textit{" Lderivada "} vamos
    a transformarlas en funciones anónimas; de esta forma al llamar las mismas, MATLAB las podrá reconocer como númericas y 
    no como variables simbólicas.
\end{justify}
\begin{figure}[H]
    \centering
    \includegraphics[scale=0.6]{anonima.PNG}        
\end{figure}
\begin{justify}
    Una vez con estos parámetros establecidos podemos olvidarnos de las derivadas y proceder a graficar ambas funciones.
\end{justify}
\begin{figure}[H]
    \centering
    \includegraphics[scale=0.5]{Grafun.PNG}        
\end{figure}

\vspace{\baselineskip}
\newpage
\begin{justify}
    En la siguiente imagen analizaremos las integrales en
    el dominio de la función \textit{" miFuncion "} junto con el área de la región acotada por la gráfica que hemos obtenido anteriormente.
\end{justify}
\begin{figure}[H]
    \centering
    \includegraphics[scale=0.6]{area.PNG}        
\end{figure}
\begin{justify}
   Es entonces cuando al obtener nuestros cálculos creamos variables simbólicas.
\end{justify}

\vspace{\baselineskip}

\begin{justify}
    En la siguiente imagen obtendremos \textbf{\(y=0\)}.
\end{justify}
\begin{figure}[H]
    \centering
    \includegraphics[scale=0.6]{raices.PNG}        
\end{figure}
\begin{justify}
    Hemos obtenido las raíces de \(x\) para conocer los valores que hacen \(x =0\) y de esta manera
    obtener \(y=0\)
\end{justify}

\vspace{\baselineskip}

\begin{justify}
 Para finalizar, obtendremos el punto de inflexión de la función que hemos venido desarrollando
 y con este resultdo cálcularemos una integral en el intervalo \(x_0 , x_1\) que acote un área determinada de nuestra función.
\end{justify}
\begin{figure}[H]
    \centering
    \includegraphics[scale=0.6]{puntosx.PNG} 
    \caption{Los valores \(t , p\) corresponden a \(x_0 , x_1 \) }       
\end{figure}

\vspace{\baselineskip}

\begin{itemize}
    \item En este punto cálcuraemos derivadas de diversas ordenes, para esto, crearemos una nueva función \textit{" granfuncion "}.
\end{itemize}
\begin{figure}[H]
    \centering
    \includegraphics[scale=0.6]{gfuncion.PNG}        
\end{figure}
\begin{justify}
    A continuación, nuestras derivadas.
\end{justify}
\begin{figure}[H]
    \centering
    \includegraphics[scale=0.6]{derivas.PNG}        
\end{figure}
\begin{justify}
    En la anterior gráfica hemos obtenido las derivadas una a una para ir definiendo las variables simbólicas como podemos apreciar.
\end{justify}

\vspace{\baselineskip}
\newpage
\begin{itemize}
    \item En el último punto vamos a verificar nuestra derivada del anterior punto de la manera \( h^3(x) + j^3(x) + k^3(x) \) para evaluar nuestro anterior resultado. 
\end{itemize}
\begin{figure}[H]
    \centering
    \includegraphics[scale=0.6]{tercerord.PNG}        
\end{figure}
\begin{justify}
    Es así, como evaluando la derivada de cada función de forma independiente, llegamos a la sumatoria de funciones derivadas de tercer orden y constatamos que obtenemos el mismo resultado que en el punto pasado.
\end{justify}

\vspace{\baselineskip}

\textbf{VISUALIZACIÓN}
\begin{itemize}
    \item Utiliza la función plot de MATLAB para visualizar las funciones definidas.
    \item Agrega etiquetas a los ejes, título y leyendas si es necesario.
    \item Experimenta con diferentes estilos de líneas, colores y marcadores para
    mejorar la visualización.
    Agrega puntos de interés como mínimos, máximos, puntos de inflexión,
etc., utilizando MATLAB para calcular estos puntos.
\end{itemize}
\vspace{\baselineskip}

\begin{justify}
    Este punto podemos resolverlo con el siguiente script.
\end{justify}

\vspace{\baselineskip}

\begin{figure}[H]
    \centering
    \includegraphics[scale=0.6]{punto3.PNG}        
\end{figure}

\begin{justify}
    Es entonces que obtenemos nuestras tres gráficas, la primera con \( y=0 \). Las otras funciones hemos modificado el color, las lineas y leyendas.
\end{justify}

\vspace{\baselineskip}
\newpage
\begin{justify}
    \textbf{Las gráficas son las siguientes:}
\end{justify}

\begin{itemize}
    \item \textbf{miFuncion}
\end{itemize}
\begin{figure}[H]
    \centering
    \includegraphics[scale=0.6]{fig1.PNG}        
\end{figure}
\begin{itemize}
    \item \textbf{Lfuncion}
\end{itemize}
\begin{figure}[H]
    \centering
    \includegraphics[scale=0.6]{fig2.PNG}        
\end{figure}

\vspace{\baselineskip}

\begin{itemize}
    \item \textbf{granfuncion}
\end{itemize}
\begin{figure}[H]
    \centering
    \includegraphics[scale=0.6]{fig3.PNG}        
\end{figure}

\vspace{\baselineskip}


\textbf{Curvas en el Plano}
\begin{itemize}
    \item Parametriza las curvas \( x^2+y^2=9 \) y \( y= x^2 \) en el intervalo \( [-4,4] \).
    \item Utiliza MATLAB para graficar ambas curvas simultáneamente utilizando
    \textit{plot}.
    \item Calcula la recta tangentes y normales a las curvas en los puntos \( x = 0\) y \( x = 1/2 \) utilize el comando diff. Representelas gráficamente, puede usar
    \textit{quiver} u otras funciones de visualización.
\end{itemize}

\vspace{\baselineskip}
\newpage
\begin{justify}
    Vamos a desarrollar los dos puntos en la siguiente imagen, mostrando el proceso de parametrización en un intervalo cerrado junto con las gráficas de la circunferencia y la parabola
\end{justify}
\begin{figure}[H]
    \centering
    \includegraphics[scale=0.5]{para.PNG}        
\end{figure}
\vspace{\baselineskip}

\begin{justify}
Ahora  trazaremos las rectas tangentes y normales en los puntos \(x = 0\) y \(x = \frac{1}{2}\) en las funciones \textit{" miFuncion "} y
\textit{" granfuncion "}.
\end{justify}

\vspace{\baselineskip}

\begin{figure}[H]
    \centering
    \includegraphics[scale=0.5]{rec.PNG}        
\end{figure}
\vspace{\baselineskip}

\begin{justify}
    Si ampliamos la gráfica podemos ver que aunque parecian solapadas varias de nuestras rectas, estás se comportan de manera paralela la una de otra.
\end{justify}

\begin{figure}[H]
    \centering
    \includegraphics[scale=0.5]{sol.PNG}        
\end{figure}
\vspace{\baselineskip}

\vspace{\baselineskip}

\begin{justify}
    \textbf{Superficies y Curvas en el espacio}
\end{justify}
\begin{itemize}
    \item Considere el paraboloide hiperbólico \(z = y^2 - x^2\) para respresentar superficies en el espacio tridimensional y represéntalo gráficamente.
    \item Utiliza funciones como \textit{surf} o \textit{mesh} para visualizar estas superficies.
    \item Paramétrice la curva interseccion del paraboloide anterior con el plano
    \(y=2\) y grafique esa curva.
    \item Calcule el plano tangente a la superficie en el punto p(1, 2, 3).
    \item Visualiza los vectores tangentes y normales utilizando quiver3 u otras
    funciones de visualización en 3D. 
\end{itemize}

\vspace{\baselineskip}

\begin{justify}
    Vamos presentar el código \textbf{donde hemos aunado todos los puntos a analizar} en tres diferentes imagenes donde hemos distribuido todos los puntos
    de manera lineal.
\end{justify}

\vspace{\baselineskip}
\newpage
\begin{justify}
    \textbf{Representación del Paraboloide Hiperbólico , trazas en z = 0, y = 0, x = 0. }
\end{justify}

\begin{figure}[H]
    \centering
    \includegraphics[scale=0.4]{fi.PNG}        
\end{figure}

\vspace{\baselineskip}

\begin{justify}
    \textbf{ Parametrización de la curva de intersección con el plano Y = 2 y  Cálculo del plano Tangente en el Punto (1, 2, 3). }
\end{justify}

\begin{figure}[H]
    \centering
    \includegraphics[scale=0.4]{se.PNG}        
\end{figure}

\vspace{\baselineskip}
\newpage
\begin{justify}
    \textbf{Visualización de Vectores Tangentes y Normales.}
\end{justify}

\begin{figure}[H]
    \centering
    \includegraphics[scale=0.4]{tri.PNG}        
\end{figure}

\vspace{\baselineskip}

\begin{justify}
    \textbf{Representación gráfica de todos los puntos}.
\end{justify}

\begin{figure}[H]
    \centering
    \includegraphics[scale=0.6]{Last.PNG}        
\end{figure}

\vspace{\baselineskip}
\newpage
\begin{justify}
    \textbf{ Análisis Adicional}.
\end{justify}
\begin{itemize}
    \item Realiza análisis adicionales según tus intereses. Por ejemplo, puedes explo-
    rar la convergencia de series infinitas, solución de ecuaciones diferenciales,
    optimización de funciones, etc.
    \item Utiliza las herramientas de MATLAB adecuadas para estos análisis, como
    \textit{sum}, \textit{ode45}, \textit{fmincon}, etc.
\end{itemize}

\vspace{\baselineskip}

\begin{justify}
    Para estos puntos opcionales hemos hecho una convergencia de serie infinita usando el comando \textit{sum}.
\end{justify}
\begin{figure}[H]
    \centering
    \includegraphics[scale=0.6]{conver.PNG}        
\end{figure}

\vspace{\baselineskip}
\newpage
\begin{justify}
    Y para finalizar, hemos desarrollado un problema de Ecuaciones Diferenciales usando el comando \textit{ode45}.
\end{justify}

\begin{figure}[H]
    \centering
    \includegraphics[scale=0.6]{aca.PNG}        
\end{figure}

Siendo entonces, este el último punto de nuestras actividades.
\begin{justify}
Dejamos para finalizar, la imagen de todo nuestro MATLAB después de haber terminado cada uno de los puntos.
\end{justify}
\begin{figure}[H]
    \centering
    \includegraphics[scale=0.3]{ñ.PNG}        
\end{figure}

Gracias.








\end{document}


